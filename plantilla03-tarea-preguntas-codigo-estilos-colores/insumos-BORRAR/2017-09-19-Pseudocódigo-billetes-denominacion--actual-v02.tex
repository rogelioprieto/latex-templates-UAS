\RequirePackage{fix-cm} %to improve performance of cm-super package when you are writing in spanish or another non-english language.



\documentclass{beamer}


%Spanish-specific commands
%--------------------------------------
\usepackage[spanish, mexico]{babel} %for proper hyphenation and translating the names of each document elements.
\def\spanishoperators{} %Mathematical commands can also be imported specifically for the Spanish language. %https://www.sharelatex.com/learn/Spanish#Reference_guide. En este caso ya está definida previamente por otros paquetes.

%--------------------------------------
%encoding
%--------------------------------------
\usepackage[utf8]{inputenc} %Allow to input letters of national alphabets directly from the keyboard. %Acentos desde el teclado
\usepackage[T1]{fontenc} %choose a font encoding which has to support specific characters for Spanish language. It's important you have installed cm-super package.
%--------------------------------------

%Set footer in all slides
%--------------------------------------
\setbeamertemplate{footline}[text line]{%
  \parbox{\linewidth}{\vspace*{-8pt}Gerardo Gálvez, Rogelio Prieto\qquad Algoritmia- FIUAS\hfill\insertpagenumber}}
%\setbeamertemplate{navigation symbols}{} %%To eliminate 
%--------------------------------------

%--------------------------------------
%tables 
%--------------------------------------
\usepackage{booktabs}

%--------------------------------------
%choose beamer theme
%--------------------------------------
\usetheme{default} %AnnArbor %CambridgeUS %Berlin


%--------------------------------------
%listings and colors package settings
%--------------------------------------
\makeatletter
 %\usepackage{color} %this line is not necessary; in beamer xcolor package is loaded by default!
 \definecolor{lightgray}{rgb}{0.95, 0.95, 0.95}
 \definecolor{darkgray}{rgb}{0.4, 0.4, 0.4}
 \definecolor{purple}{rgb}{0.65, 0.12, 0.82}
 \definecolor{editorGray}{rgb}{0.95, 0.95, 0.95}
 \definecolor{editorOcher}{rgb}{1, 0.5, 0} % #FF7F00 -> rgb(239, 169, 0)
 \definecolor{editorGreen}{rgb}{0, 0.5, 0} % #007C00 -> rgb(0, 124, 0)
 %\usepackage{upquote}
 \usepackage{listings}
 %\usepackage{listingsutf8}
 
 %define algoritmia language
 \lstdefinelanguage{algoritmialanguage}{
 keywords={INICIO,FIN,IMPRIMIR,LEER,SI,SI_NO,FIN_SI,SEGUN_SEA,CASO,FIN_CASO,FIN_SEGUN_SEA,ENTONCES,MIENTRAS,REPETIR,PARA,POW,SQRT},
 keywordstyle=\color{blue}\bfseries,
 ndkeywords={CONST,ENTERO,REAL,BOOLEAN,CADENA,CARACTER},
 ndkeywordstyle=\color{magenta}\bfseries, %\color{darkgray}\bfseries, %\color{purple}\bfseries,
 identifiers={},
 identifierstyle= \color{black},%\color{darkgray},
 sensitive=true,
 comment=[l]{//},
 morecomment=[s]{/*}{*/},
 commentstyle=\color{editorGreen}\ttfamily, %\color{darkgray}\ttfamily,	%commentstyle=\color{purple}\ttfamily,
 stringstyle=\color{red}\ttfamily,
 morestring=[b]',
 morestring=[b]".
 }
 
 
 
 %define colors for coding
 \definecolor{codegreen}{rgb}{0,0.6,0}
 \definecolor{codegray}{rgb}{0.5,0.5,0.5}
 \definecolor{codepurple}{rgb}{0.58,0,0.82}
 \definecolor{backcolour}{rgb}{0.95,0.95,0.92}
 \lstdefinestyle{mycodestyle}{
     backgroundcolor=\color{backcolour},   
     commentstyle=\color{codegreen},
     keywordstyle=\color{magenta},
     numberstyle=\tiny\color{codegray},
     stringstyle=\color{codepurple},
     columns 		= fullflexible,
     basicstyle= \tiny\ttfamily, %\footnotesize\ttfamily,
     breakatwhitespace=false,         
     breaklines=true,                 
     %captionpos=b,                    
     keepspaces=true,                 
     numbers=left,                    
     numbersep=5pt,                  
     showspaces=false,                
     showstringspaces=false,
     showtabs=false,                  
     tabsize=3,
     frame=lines,
     %directivestyle=\color{black}
              %xleftmargin=17pt,
              %framexleftmargin=17pt,
              %framexrightmargin=5pt,
              %framexbottommargin=4pt   
 }
 %using mycodestyle
 %\lstset{style=mycodestyle}
 \lstset{
 	style=mycodestyle,
    inputencoding = utf8,  % Input encoding
    extendedchars = true,  % Extended ASCII
    texcl         = false,  % Activate LaTeX commands in comments  %%true
    mathescape    = false,   % Mathematical expressions between $  %%true
    captionpos    = b,     % Caption position
    literate=			   % Support additional characters
      {á}{{\'a}}1 {é}{{\'e}}1 {í}{{\'i}}1 {ó}{{\'o}}1 {ú}{{\'u}}1
      {Á}{{\'A}}1 {É}{{\'E}}1 {Í}{{\'I}}1 {Ó}{{\'O}}1 {Ú}{{\'U}}1
      {à}{{\`a}}1 {è}{{\`e}}1 {ì}{{\`i}}1 {ò}{{\`o}}1 {ù}{{\`u}}1
      {À}{{\`A}}1 {È}{{\'E}}1 {Ì}{{\`I}}1 {Ò}{{\`O}}1 {Ù}{{\`U}}1
      {ä}{{\"a}}1 {ë}{{\"e}}1 {ï}{{\"i}}1 {ö}{{\"o}}1 {ü}{{\"u}}1
      {Ä}{{\"A}}1 {Ë}{{\"E}}1 {Ï}{{\"I}}1 {Ö}{{\"O}}1 {Ü}{{\"U}}1
      {â}{{\^a}}1 {ê}{{\^e}}1 {î}{{\^i}}1 {ô}{{\^o}}1 {û}{{\^u}}1
      {Â}{{\^A}}1 {Ê}{{\^E}}1 {Î}{{\^I}}1 {Ô}{{\^O}}1 {Û}{{\^U}}1
      {œ}{{\oe}}1 {Œ}{{\OE}}1 {æ}{{\ae}}1 {Æ}{{\AE}}1 {ß}{{\ss}}1
      {ű}{{\H{u}}}1 {Ű}{{\H{U}}}1 {ő}{{\H{o}}}1 {Ő}{{\H{O}}}1
      {ç}{{\c c}}1 {Ç}{{\c C}}1 {ø}{{\o}}1 {å}{{\r a}}1 {Å}{{\r A}}1
      {€}{{\euro}}1 {£}{{\pounds}}1 {«}{{\guillemotleft}}1
      {»}{{\guillemotright}}1 {ñ}{{\~n}}1 {Ñ}{{\~N}}1 {¿}{{?`}}1
      {¿}{{?`}}1 {¡}{{!`}}1
      % ¿ and ¡ are not correctly displayed if inconsolata font is used
      % together with the lstlisting environment. Consider typing code in
      % external files and using \lstinputlisting to display them instead.
  }
 


 \makeatother


%--------------------------------------
%patch to fix showing tabs using lstlistings in beamer.
%--------------------------------------
\makeatletter
\def\beamer@verbatimreadframe{%
  \begingroup%
  \let\do\beamer@makeinnocent\dospecials%
  \count@=127%
  \@whilenum\count@<255 \do{%
    \advance\count@ by 1%
    \catcode\count@=11%
  }%
  \beamer@makeinnocent\^^L% and whatever other special cases
  \beamer@makeinnocent\^^I% <-- PATCH: allows tabs to be written to temp file
  \endlinechar`\^^M \catcode`\^^M=12%
  \@ifnextchar\bgroup{\afterassignment\beamer@specialprocessframefirstline\let\beamer@temp=}{\beamer@processframefirstline}}%
\makeatother






%Information to be included in the title page:
\title{Algoritmia}
\author{Gerardo Gálvez Gámez, Rogelio Prieto Alvarado}
\institute{Facultad de Informática Culiacán\\Universidad Autónoma de Sinaloa}
\date{2017}
\begin{document}
\maketitle

%--------------------------------------
%fragile option
%--------------------------------------
%NOTE: every frame has to use the fragile option.
%If a frame contains fragile text, different internal mechanisms are used to typeset the frame to ensure that inside the frame the character codes can be reset. The price of switching to another internal mechanism is that either you cannot use overlays or an external file needs to be written and read back (which is not always desirable).

\begin{frame}
\frametitle{Descripción del problema}
Juan necesita pagar una determinada cantidad de dinero en efectivo, para ello utilizará billetes de \$200 y de \$100. Elabore un algoritmo que dada la cantidad a pagar determine la cantidad de billetes requeridos de cada denominación. En caso de existir un sobrante el algoritmo deberá emitir un mensaje; de igual manera si la cantidad a pagar es inferior a la menor denominación de billetes.
\end{frame}

\begin{frame}
\frametitle{Análisis del problema}
\begin{itemize}
\item Información de salida
\\¿?
\item Datos Conocidos
\\¿?
\item Datos de Entrada
\\¿?
\item Variables de trabajo
\\¿?
\item Restricciones
\\¿?
\end{itemize}
\end{frame}

\begin{frame}[fragile] %fragile option 
%\frametitle{Construcción del pseudocódigo (1 de 2)}

\begin{lstlisting}[language=algoritmialanguage]
	/Objetivo: Determinar el Total de Producción de Mangos Finos
	//Programador: MC. Gerardo Gálvez G.
	//Fecha:24/Septiembre/2021
	
	INICIO
	//Definición de Constantes y Variables 
	CONST ENTERO  ProduccionPlantaFina=8500
	ENTERO NumeroArboles,TotalProduccion,NumeroPlantasFinas	
	//Lecturas de Datos de Entrada
	IMPRIMIR "Teclee el Número de Árboles en la Huerta:"
	LEER NumeroArboles
	//Procesamiento de los Datos
	NumeroPlantasFinas = (NumeroArboles - NumeroArboles MOD 2 ) / 2  
	TotalProduccion = NumeroPlantasFinas* ProduccionPlantaFina
	//Impresión de Resultados
	IMPRIMIR "Total de Mangos Finos Producidos: ", TotalProduccion
	FIN
\end{lstlisting

\begin{lstlisting}[language=algoritmialanguage]
//Objetivo: Determinar cuantos billetes necesitamos de cada denominación
//Programador: Alejandra Preciado
//Fecha: 19/09/2017
INICIO
	//Definición de constantes y variables
	CONST ENTERO Billete1 = 200 , Billete2 = 100 , NoExiste = 0
	REAL CantidadPagar , Cantidad1 , Cantidad2
	ENTERO CantidadBilletes1 , CantidadBilletes2 
	//Lectura de datos de entrada
	IMPRIMIR "El precio del producto es: $"
	LEER CantidadPagar
	//PROCESO
	SI CantidadPagar >= Billete1 ENTONCES
		CantidadBilletes= CantidadPagar / Billete1
		Cantidad1= CantidadBilletes * Billete1
		Cantidad2= CantidadPagar - Cantidad1
		SI Cantidad2 >= Billete2 ENTONCES
			CantidadBilletes2= Cantidad2 / Billete2
		SI_NO
			CantidadBilletes2= NoExiste /////RPA: ERROR ES VARIABLE ENTERA, NO ES POSIBLE ASIGNARLE ESE VALOR
		FIN_SI
	SI_NO   
		SI CantidadPagar >= Billete2 ENTONCES
			CantidadBilletes2= CantidadPagar / Billete1 //RPA: PREGUNTA: ¿SE DIVIDE ENTRE Billete2?
		SI_NO
			SI CantidadPagar < Billete2 ENTONCES  //RPA: PREGUNTA: ¿ES NECESARIA ESTA PREGUNTA? 
				CantidadBilletes2= NoExiste ////RPA: ERROR ES VARIABLE ENTERA, NO ES POSIBLE ASIGNARLE ESE VALOR
				CantidadBilletes1= NoExiste ////RPA: ERROR ES VARIABLE ENTERA, NO ES POSIBLE ASIGNARLE ESE VALOR
			FIN_SI ////RPA: FALTARON AQUI MÁS FIN_SI
		FIN_SI ////RPA: FUE AGREGADO
	FIN_SI ////RPA: FUE AGREGADO
\end{lstlisting}
\end{frame}


\begin{frame}[fragile] %fragile option 
\frametitle{Construcción del pseudocódigo (2 de 2)}
\begin{lstlisting}[language=algoritmialanguage,firstnumber=32]
   //continuación...
   //Impresión de resultados
   IMPRIMIR "La cantidad de billetes de 200 es: " CantidadBilletes1
   IMPRIMIR "La cantidad de billetes de 100 es: " CantidadBilletes2
   IMPRIMIR "El precio no alcanza al billete: " NoExiste ////RPA: FALTA LA SEPARACIÓN, ADEMÁS UNA PREGUNTA: ¿SIEMPRE SE IMPRIME LA CADENA NO EXISTE? ¿APLICA ESTO PARA TODOS LOS CASOS?
FIN 
\end{lstlisting}
\end{frame}





%\begin{frame}
%\frametitle{First Slide}
%Contents of the first slide
%\end{frame}
\end{document}