\RequirePackage{fix-cm} %to improve performance of cm-super package when you are writing in spanish or another non-english language.
\documentclass[spanish,11pt,twoside]{article}

%--------------------------------------
%Spanish-specific commands
%--------------------------------------
\usepackage[spanish, mexico]{babel} %for proper hyphenation and translating the names of each document elements.
\def\spanishoperators{} %Mathematical commands can also be imported specifically for the Spanish language. %https://www.sharelatex.com/learn/Spanish#Reference_guide. En este caso ya está definida previamente por otros paquetes.

%--------------------------------------
%encoding
%--------------------------------------
\usepackage[utf8]{inputenc} %Allow to input letters of national alphabets directly from the keyboard.
\usepackage[T1]{fontenc} %choose a font encoding which has to support specific characters for Spanish language


%--------------------------------------
%Colors
%--------------------------------------
%Using this package, you can set the font color, text background, or page background. You can choose from predefined colors or define your own colors using RGB, Hex, or CMYK. Mathematical formulas can also be colored.
\usepackage[usenames,dvipsnames,svgnames,table]{xcolor} %allow predefined name colors
%\usepackage[usenames,dvipsnames]{xcolor} %allow predefined 
%\usepackage[dvipsnames]{xcolor} %allow predefined
%TO-DO url:https://www.sharelatex.com/learn/Using_colours_in_LaTeX#/Reference_guide
%TO-DO url:https://en.wikibooks.org/wiki/LaTeX/Colors

\definecolor{uasblue}{HTML}{19407A}



%--------------------------------------
%Table packages and style commands 
%--------------------------------------
%------------style for color lines------
\usepackage[table]{xcolor} %Load the xcolor package with the tables option for table support
%disable, we'll move before table environment:
%\rowcolors{2}{gray!15}{white} %Declare alternating row colors
\newcommand{\head}[1]{ %Define a macro for the table header appearance:
	\textcolor{white}{\textbf{#1}}}

\renewcommand{\arraystretch}{1.5} %Enlarge the default tabular line spacing

%------------style for clean and better legibility tables--------
\usepackage{booktabs} %has been written with good design in mind. %Specifically, it enhances lines in tables. It focuses on horizontal lines with improved spacing and adjustable thickness.
\usepackage{float} %for use H option in tables: force position "HERE" 


%--------------------------------------
%blocks around text to inform or alert
%--------------------------------------
%drawing admonition blocks around text to inform or alert
%your readers about something particular. The specific aim of this package is to use
%FontAwesome 5 icons to ease the illustration of these boxes.
\usepackage{awesomebox}
%\usepackage{fontawesome5}

\newcommand{\rocketbox}[1]{
	\awesomebox[uasblue]{2pt}{\faRocket}{uasblue}{#1}
}
\newcommand{\myawesomebox}[2]{ %receive icon and content(text)
	\awesomebox[uasblue]{2pt}{#1}{uasblue}{#2}
}







\usepackage{amsmath}
%--------------------------------------
%add specific information
%--------------------------------------
%TODO Change document information
\newcommand{\myuniversity}{Universidad Autónoma de Sinaloa}
\newcommand{\myfaculty}{Facultad de Informática Culiacán}
\newcommand{\myprofs}{\{Rogelio Prieto Alvarado, Gerardo Gálvez Gámez\}}
\newcommand{\mycourse}{Algoritmia}
\newcommand{\mysubject}{5. Resolución de problemas de selección}
\newcommand{\mydate}{Diciembre 10, 2021}
\newcommand{\myhandoutnumber}{06} %Número de tarea
\newcommand{\mytitledoc}{Problemas de Selección simple, doble o múltiple}


\newlength{\toppush}
\setlength{\toppush}{2\headheight}
\addtolength{\toppush}{\headsep}

\newcommand{\htitle}[3]{\noindent\vspace*{-\toppush}\newline\parbox{6.5in}
{
	\begin{wrapfigure}{l}{1.3cm}
	%\caption{A wrapped figure going nicely inside the text.}\label{wrap-fig:1}
	\vspace*{-2ex}\includegraphics[width=1.3cm]{assets/uasbn}
	\end{wrapfigure} 
	\textit{\mycourse: \mysubject}\hfill\newline
\myuniversity\newline \myfaculty \hfill #3\newline
\myprofs\hfill \textcolor{uasblue}{\textbf{Tarea #1}} \newline \vspace*{-3.5ex}\newline
\mbox{}\hrulefill\mbox{}}\vspace*{1ex}\mbox{}\newline
\begin{center}{\Large\bf \textcolor{uasblue}{#2}}\end{center}}

\newcommand{\handout}[3]{\thispagestyle{empty}
 \markboth{Tarea #1: #2}{Tarea #1: #2}
  \pagestyle{myheadings}\htitle{#1}{#2}{#3}}

\setlength{\oddsidemargin}{0pt}
\setlength{\evensidemargin}{0pt}
\setlength{\textwidth}{6.5in}
\setlength{\topmargin}{0in}
\setlength{\textheight}{8.5in}

%--------------------------------------
%wrap text around a picture
%--------------------------------------
\usepackage{graphicx,wrapfig}




%--------------------------------------	
%crear nuevas variables para ser agregadas en los metadatos del PDF.
%--------------------------------------
%\makeatletter
%\let\pdftitle\@title
%\let\pdfauthor\@author
%\makeatother
\let\pdftitle\mycourse
\let\pdfauthor\myprofs
\let\pdfsubject\mysubject

%--------------------------------------
%hyperreferences
%--------------------------------------
%\usepackage{color}
\usepackage{hyperref} %is used to han­dle cross-ref­er­enc­ing com­mands in LaTeX to pro­duce hy­per­text links in the doc­u­ment.
\hypersetup{
	colorlinks=true, %set true if you want colored links
	%allcolors=blue  %set all colors options (without border and field options)
	linktoc=all,     %set to all if you want both sections and subsections linked
	linkcolor= uasblue, %NavyBlue, %black,  %choose some color if you want internal links to stand out 
	citecolor= blue, % blue,   %color for bibliographical citations in text.
	urlcolor=uasblue,%NavyBlue,     %color for URL links         
	%--------------------------------------
	%Configuración del PDF resultante
	%--------------------------------------
	pdfauthor ={\pdfauthor}, %, \pdfsupervisor}, 
	pdftitle ={\pdftitle\ - Tarea \myhandoutnumber},
	pdfsubject ={\pdfsubject} ,
	pdfkeywords ={Algoritmos,Programación,Informática,UAS,Licenciatura},
	%pdfstartview ={ FitV } ,
	%pdfview ={ FitH } ,
	pdfpagemode ={ FullScreen },
	backref ={ true },
	bookmarks={true},
	%pdfpagelabels,
	bookmarksnumbered=true
}



%--------------------------------------
%listings and colors package settings
%--------------------------------------
\makeatletter
%\usepackage{color} %this line is not necessary; in beamer xcolor package is loaded by default!
%\definecolor{lightgray}{rgb}{0.95, 0.95, 0.95}
%\definecolor{darkgray}{rgb}{0.4, 0.4, 0.4}
%\definecolor{purple}{rgb}{0.65, 0.12, 0.82}
%\definecolor{editorGray}{rgb}{0.95, 0.95, 0.95}
%\definecolor{editorOcher}{rgb}{1, 0.5, 0} % #FF7F00 -> rgb(239, 169, 0)
%\definecolor{editorGreen}{rgb}{0, 0.5, 0} % #007C00 -> rgb(0, 124, 0)
%\usepackage{upquote}
\usepackage{listings}
%\usepackage{listingsutf8}


%define algoritmia language
\lstdefinelanguage{algoritmialanguage}{
	keywords={INICIO,FIN,IMPRIMIR,LEER,SI,SI_NO,FIN_SI,SEGUN_SEA,CASO,FIN_CASO,FIN_SEGUN_SEA,ENTONCES,MIENTRAS,REPETIR,PARA,POW,SQRT,MOD},
	%keywordstyle=\color{blue}\bfseries,
	%ndkeywords={CONST,ENTERO,REAL,BOOLEAN,CADENA,CARACTER},
	%ndkeywordstyle=\color{magenta}\bfseries, %\color{darkgray}\bfseries, %\color{purple}\bfseries,
	keywords=[2]{CONST,ENTERO,REAL,BOOLEAN,CADENA,CARACTER},
	%keywordstyle=[2]\color{magenta}\bfseries, %\color{darkgray}\bfseries,	%\color{purple}\bfseries,
	%identifiers={Texto},
	%identifierstyle= \color{black}%\color{darkgray},
	sensitive=true,
	comment=[l]{//},
	morecomment=[s]{/*}{*/},
	%commentstyle=\color{editorGreen}\ttfamily, %\color{darkgray}\ttfamily, %commentstyle=\color{purple}\ttfamily,
	%stringstyle=\color{orange}\ttfamily,
	morestring=[b]',
	morestring=[b]"
}



%define colors for coding
\definecolor{codegreen}{rgb}{0,0.6,0} %for comments
\definecolor{codegray}{rgb}{0.5,0.5,0.5} %for number lines
\definecolor{codepurple}{rgb}{0.58,0,0.82} %for strings
\definecolor{codecardinal}{HTML}{800000} %for strings
\definecolor{backcolour}{rgb}{0.95,0.95,0.92} %background box color
\definecolor{codeblue}{HTML}{00008B}%#5D8EB5

\lstdefinestyle{mycodestyle}{
	language=algoritmialanguage,
	backgroundcolor=\color{backcolour},   
	commentstyle=\color{codegreen},
	%commentstyle=\color{editorGreen}\ttfamily, %\color{darkgray}\ttfamily,
	keywordstyle=\color{codeblue},%\color{magenta},%RoyalBlue
	keywordstyle=[2]\color{magenta}\bfseries, %\color{darkgray}\bfseries,	%\color{purple}\bfseries,
	numberstyle=\tiny\color{codegray},
	stringstyle=\color{codepurple},%\color{orange},\color{editorOcher},codecardinal
	columns= fullflexible,
	basicstyle= \small\ttfamily, %\footnotesize\ttfamily, %\tiny\ttfamily,
	breakatwhitespace=false,         
	breaklines=true,                 
	%captionpos=b,                    
	keepspaces=true,                 
	numbers=left,                    
	numbersep=5pt,                  
	showspaces=false,                
	showstringspaces=false,
	showtabs=false,                  
	tabsize=3,
	frame=lines,
	%directivestyle=\color{black}
	%xleftmargin=17pt,
	%framexleftmargin=17pt,
	%framexrightmargin=5pt,
	%framexbottommargin=4pt   
}
%using mycodestyle
%\lstset{style=mycodestyle}
\lstset{
	style=mycodestyle,
	inputencoding = utf8,  % Input encoding
	extendedchars = true,  % Extended ASCII
	texcl         = false,  % Activate LaTeX commands in comments  %%true
	mathescape    = false,   % Mathematical expressions between $  %%true
	captionpos    = b,     % Caption position
	literate=			   % Support additional characters
	{á}{{\'a}}1 {é}{{\'e}}1 {í}{{\'i}}1 {ó}{{\'o}}1 {ú}{{\'u}}1
	{Á}{{\'A}}1 {É}{{\'E}}1 {Í}{{\'I}}1 {Ó}{{\'O}}1 {Ú}{{\'U}}1
	{à}{{\`a}}1 {è}{{\`e}}1 {ì}{{\`i}}1 {ò}{{\`o}}1 {ù}{{\`u}}1
	{À}{{\`A}}1 {È}{{\'E}}1 {Ì}{{\`I}}1 {Ò}{{\`O}}1 {Ù}{{\`U}}1
	{ä}{{\"a}}1 {ë}{{\"e}}1 {ï}{{\"i}}1 {ö}{{\"o}}1 {ü}{{\"u}}1
	{Ä}{{\"A}}1 {Ë}{{\"E}}1 {Ï}{{\"I}}1 {Ö}{{\"O}}1 {Ü}{{\"U}}1
	{â}{{\^a}}1 {ê}{{\^e}}1 {î}{{\^i}}1 {ô}{{\^o}}1 {û}{{\^u}}1
	{Â}{{\^A}}1 {Ê}{{\^E}}1 {Î}{{\^I}}1 {Ô}{{\^O}}1 {Û}{{\^U}}1
	{œ}{{\oe}}1 {Œ}{{\OE}}1 {æ}{{\ae}}1 {Æ}{{\AE}}1 {ß}{{\ss}}1
	{ű}{{\H{u}}}1 {Ű}{{\H{U}}}1 {ő}{{\H{o}}}1 {Ő}{{\H{O}}}1
	{ç}{{\c c}}1 {Ç}{{\c C}}1 {ø}{{\o}}1 {å}{{\r a}}1 {Å}{{\r A}}1
	{€}{{\euro}}1 {£}{{\pounds}}1 {«}{{\guillemotleft}}1
	{»}{{\guillemotright}}1 {ñ}{{\~n}}1 {Ñ}{{\~N}}1 {¿}{{?`}}1
	{¿}{{?`}}1 {¡}{{!`}}1
	% ¿ and ¡ are not correctly displayed if inconsolata font is used
	% together with the lstlisting environment. Consider typing code in
	% external files and using \lstinputlisting to display them instead.
}
\makeatother



%--------------------------------------
%patch to fix showing tabs using lstlistings in beamer.
%--------------------------------------
\makeatletter
\def\beamer@verbatimreadframe{%
	\begingroup%
	\let\do\beamer@makeinnocent\dospecials%
	\count@=127%
	\@whilenum\count@<255 \do{%
		\advance\count@ by 1%
		\catcode\count@=11%
	}%
	\beamer@makeinnocent\^^L% and whatever other special cases
	\beamer@makeinnocent\^^I% <-- PATCH: allows tabs to be written to temp file
	\endlinechar`\^^M \catcode`\^^M=12%
	\@ifnextchar\bgroup{\afterassignment\beamer@specialprocessframefirstline\let\beamer@temp=}{\beamer@processframefirstline}}%
\makeatother



\begin{document}


\handout{\myhandoutnumber}{\mytitledoc}{\mydate}
\setlength{\parindent}{0pt}

\newcommand{\solution}{
  \medskip
  {\bf Solution:}
}

%TODO Change content and questions
\textbf{Objetivo:} El alumno demostrará la habilidad alcanzada en clases, para analizar, construir y probar pseudocódigos de diversos problemas, utilizando procedimientos con estructuras de selección simple, doble, doble anidada o múltiple.\\
\textbf{Fecha de entrega:} Especificada en el aula virtual.\\
\textbf{Instrucciones:} Resuelva el problema 1 (obligatorio). De los restantes, elija y resuelva cuatro. Los algoritmos deberán ser resueltos según la metodología vista en clase que incluye:
a. Análisis del problema
b. Construcción del algoritmo en pseudocódigo.
c. Verificación (prueba y depuración)\\
\textbf{Instrucciones de entrega:} Deberá resolver el ejercicio y realizar las respuestas a mano.

\medskip

\hrulefill

\section*{Descripción de problemas}


\begin{enumerate}
	\item Para las soluciones a los problemas resueltos en la diapositiva \texttt{Estructuras Algoritmicas de Seleccion simples.pdf}, resolver:
	\begin{enumerate}
		\item Indique ¿cuáles problemas pueden ser resueltos usando estructura algorítmica de selección Doble?
		\item Elija uno de los problemas que indicó en la pregunta 1, reescriba la «construcción del algoritmo» utilizando estructura de selección doble.
	\end{enumerate}
	

	\item Elaborar un pseudocódigo que permita leer 3 números enteros y determine cuál es mayor. Considere que los tres números son diferentes.	
	
	
	\vspace{24pt}

	Del Libro "Metodología de la programación"\ de Osvaldo Cairo, resuelva el siguiente
	problema:
	
	
	
	\item PS 2.18, página 415. Construya un algoritmo que le permita calcular e imprimir el costo final de una llamada telefónica. Para calcular el costo final se sigue lo indicado en la siguiente tabla:
	
	%Example, source by Stefan Kottwitz in LaTeX Cookbook 1st edition, chapter 6.
	%NOTE: this table only uses table package.
	\begin{table}[H]%[ht]
		\rowcolors{2}{gray!15}{white} %Declare alternating row colors
		\renewcommand{\arraystretch}{1.5} %Enlarge the default tabular line spacing
		\centering\footnotesize%\small
		\sffamily%\ttfamily
		
		%\caption{Texto de caption, debe aparecer antes de la tabla}
		\label{table:problemas:llamada}
		\begin{tabular}{lccc}
			\rowcolor{uasblue}%black!75
			\head{Clave}& \head{Zona} 		& \head{Precio/Minuto (3 primeros)} & \head{Precio/Minuto (del 4to en adelante)}\\
			12 			& América del Norte	& 2									& 1.5 \\
			15			& América Central	& 2.2								& 1.8 \\
			18			& América del Sur	& 4.5								& 3.5 \\
			19			& Europa			& 3.5								& 2.7 \\
			23			& Asia				& 6									& 4.6 \\
			25			& África			& 6									& 4.6 \\
			29			& Oceanía			& 5									& 3.9 \\
		\end{tabular}
	\end{table}
	
\end{enumerate}


\vspace{120px}
\newpage
Ejemplo de pseudocódigo:
\begin{lstlisting}%[language=algoritmialanguage]
	
	//Objetivo:  Calcular el agua restante en el tanque 
	//Programador: Rogelio Prieto Alvarado
	//Fecha:10/diciembre/2021
	INICIO
		//Declaración de constantes y variables
		CONST ENTERO ConsumoSemanal=183
		CONST ENTERO CapacidadInicial=10000
		ENTERO CantidadRestante, Semana
		//Lectura de datos de entrada
		//Procesamiento de datos e impresión de resultados
		//Paso 1. Inicializar la variable de control
		CantidadRestante=CapacidadInicial
		Semana=1
		MIENTRAS(CantidadRestante>=ConsumoSemanal)
			CantidadRestante=CantidadRestante-ConsumoSemanal
			IMPRIMIR "Semana ",Semana," quedan ",CantidadRestante," litros en el tanque"
			Semana=Semana+1
		FIN_MIENTRAS
	FIN
\end{lstlisting}


Ejemplo de programa en lenguaje c:
\begin{lstlisting}[language=C]
//Objetivo:  Calcular el agua restante en el tanque 
//Programador: Rogelio Prieto Alvarado
//Fecha:10/diciembre/2021
	
#include <stdio.h>

int main(void)
{
	//Declaración de constantes y variables
	const int ConsumoSemanal=183;
	const int CapacidadInicial=10000;
	int CantidadRestante, Semana;
	//Lectura de datos de entrada
	//Procesamiento de datos e impresión de resultados
	//Paso 1. Inicializar la variable de control
	CantidadRestante=CapacidadInicial;
	Semana=1;
	while(CantidadRestante>=ConsumoSemanal){
		CantidadRestante=CantidadRestante-ConsumoSemanal;
		printf("Semana %d quedan %d litros en el tanque\n", Semana, CantidadRestante);
		Semana=Semana+1;
	}
}
\end{lstlisting}



\begin{noteblock} %tipblock 
	\textcolor{uasblue}{\textbf{\textsf{Nota.}}} Recuerde que debe elegir la estructura de selección que le permita construir la solución más eficiente posible. 
	
\end{noteblock}

%\rocketbox{	\textcolor{uasblue}{\textbf{\textsf{Nota.}}} Texto }


\vspace{120px}



%--------------------------------------
% Insert bibligraphy
%--------------------------------------
%change References heading
%\renewcommand{\bibname}{whatever} %used for book document class
\renewcommand{\refname}{Bibliografía} %used for article document class
%\addto\captionsspanish{\renewcommand{\refname}{loquesea2}}
%\addto\captionsspanish{\renewcommand{\bibname}{loquesea1}}



\nocite{battistutti2005metodologia} %makes one cite entrie.
\nocite{*} %NOTE: only works if you cite almost one key before.
%Define style and insert bibliography
\bibliographystyle{plainnat} %sets style you can use: ieeetr acmtrans apalike  plainnat
\bibliography{assets/books-algoritmia} %booksIA.bib is the file containing the bibliography entries.

%----------------------------------------------------------------




\end{document}
